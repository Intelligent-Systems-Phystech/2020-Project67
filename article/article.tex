\documentclass[12pt, twoside]{article}
\usepackage{jmlda}
\newcommand{\hdir}{.}

\begin{document}

\title
    %[Шаблон статьи для публикации] % краткое название; не нужно, если полное название влезает в~колонтитул
    {Отбор тем в тематических моделях для разведочного информационного поиска.}
\author
    [Хоменко~Р.\,Д.] % список авторов (не более трех) для колонтитула; не нужен, если основной список влезает в колонтитул
    {Хоменко~Р.\,Д., Воронцов~К.\,В., Янина~А.\,О.} % основной список авторов, выводимый в оглавление
    [Хоменко~Р.\,Д.$^{1}$, Воронцов~К.\,В.$^{1}$, А.\,О.~Янина$^{1}$] % список авторов, выводимый в заголовок; не нужен, если он не отличается от основного
\email
    {r.d.khomenko@yandex.ru; vokov@forecsys.ru;  yanina.anastasia.mipt@gmail.com}
\thanks
    %{Работа выполнена при
     %частичной
     %финансовой поддержке РФФИ, проекты \No\ \No 00-00-00000 и 00-00-00001.}
    {Задачу поставил: Воронцов~К.\,В.
        Консультант: Янина~А.\,О.}
\organization
    {$^1$ Московский физико-технический институт, Москва, Россия}%; $^2$Организация}
\abstract
    {Одним из приложений тематического моделирования является разведочный информационный
    поиск. Данный метод поиска позволяет пользователю получить набор релевантных
    документов, ознакомление с которым поможет сформировать понимание некоторой
    предметной области.
    
    В данной работе исследуется методы отбора тем для повышения качества разведочного
    информационного поиска. Используются следующие методы отбора тем: метод
    рекурсивного исключения признаков,
    использование фоновых и предметных тем в тематических моделях,
    отбор тем по критерию чистоты лексического ядра.

    Тут должно быть описание результатов работы.

\bigskip
\noindent
\textbf{Ключевые слова}: \emph {
    разведочный информационный поиск;
    тематическое моделирование;
    аддитивная регуляризация;
    методы отбора тем;
    рекурсивное извлечение признаков
    }
}

\titleEng
	[JMLDA paper template] % краткое название; не нужно, если полное название влезает в~колонтитул
    {Machine Learning and Data Analysis journal paper template}
\authorEng
	[F.\,S.~Author] % список авторов (не более трех) для колонтитула; не нужен, если основной список влезает в колонтитул
	{F.\,S.~Author, F.\,S.~Co-Author, and F.\,S.~Name} % основной список авторов, выводимый в оглавление
    [F.\,S.~Author$^1$, F.\,S.~Co-Author$^2$, and F.\,S.~Name$^{1, 2}$] % список авторов, выводимый в заголовок; не нужен, если он не отличается от основного
\thanksEng
    {The research was
     %partially
    	 supported by the Russian Foundation for Basic Research (grants 00-00-0000 and 00-00-00001).
    }
\organizationEng
    {$^1$Organization, address; $^2$Organization, address}
\abstractEng
    {This is the template of the paper submitted to the journal ``Machine Learning and Data Analysis''.
		
	\noindent
	The title should be concise and informative. Titles are often used in information-retrieval systems. Avoid abbreviations and formulae where possible.
	Please clearly indicate the last names and initials of each author and check that all names are accurately spelled. Present the authors' affiliation
	addresses where the actual work was done.
	Provide the full postal address of each affiliation, including the country name and, if available, the
	e-mail address of each author.
	Provide only institutional affiliation, department/division affiliation are not required.

	\noindent
	A concise and factual abstract is required.
	The purpose of the abstract is to provide a summary~of the paper enabling the reader to decide whether or not to read the full text.
    	The abstract should state briefly the purpose of the research, the principal results and major conclusions.
    	An abstract is often presented separately from the article, so it must be able to stand alone.
    	For this reason, References should be avoided, but if essential, then cite the author(s) and year(s).
    	Also, non-standard or uncommon abbreviations should be avoided, but if essential they must be defined at their first mention in the abstract itself.
    	The requirements on the size of the abstract is about 200--300 words.
    	It should be provided in the next structured manner:
	
	\noindent
	\textbf{Background}:	One paragraph about the problem, existent approaches and its limitations.
	
	\noindent
	\textbf{Methods}: One paragraph about proposed method and its novelty.
	
	\noindent
	\textbf{Results}: One paragraph about major properties of the proposed method and experiment results if applicable.
	
	\noindent
	\textbf{Concluding Remarks}: One paragraph about the place of the proposed method among existent approaches.
		
	\noindent
	Immediately after the abstract, provide 5-7 keywords, avoiding general and plural terms and multiple concepts (avoid, for example, ``and'', ``of'').
	Use keywords that are specific and that reflect what is essential about the paper.
	Use keywords from the abstract, introduction and conclusion.
	These keywords will be used for indexing purposes.
		
	\noindent
    	\textbf{Keywords}: \emph{keyword; keyword; more keywords, separated by ``;''}}

%данные поля заполняются редакцией журнала
%\doi{10.21469/22233792}
%\receivedRus{01.01.2017}
%\receivedEng{January 01, 2017}

\maketitle
%\linenumbers

\section{Введение}
Тут будет введение и ссылки на литературу \cite{vorontsov_co/survey-artm},
\cite{vorontsov_co/non-bayesian-armt}, \cite{yanina/multiobjective-topic-modeling},
\cite{vorontsov_co/topic-selection-artm}, \cite{vorontsov_co/artm}.

\section{Постановка задачи}
Тут будет постановка задачи.

\paragraph{Название параграфа}
Разделы и~параграфы, за исключением списков литературы, нумеруются.

\section{Заключение}
Желательно, чтобы этот раздел был, причём он не~должен дословно повторять аннотацию.
Обычно здесь отмечают, каких результатов удалось добиться, какие проблемы остались открытыми.

\bibliographystyle{plain}
\bibliography{lit}

\end{document}
