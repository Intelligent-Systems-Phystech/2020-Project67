\documentclass[12pt, twoside]{article}
\usepackage{jmlda}
\newcommand{\hdir}{.}

\begin{document}

\title
    [Шаблон статьи для публикации] % краткое название; не нужно, если полное название влезает в~колонтитул
    {Шаблон статьи для публикации в журнале <<Машинное обучение и анализ данных>>}
\author
    [И.\,О.~Автор] % список авторов (не более трех) для колонтитула; не нужен, если основной список влезает в колонтитул
    {И.\,О.~Автор, И.\,О.~Соавтор, И.\,О.~Фамилия} % основной список авторов, выводимый в оглавление
    [И.\,О.~Автор$^1$, И.\,О.~Соавтор$^2$, И.\,О.~Фамилия$^{1,2}$] % список авторов, выводимый в заголовок; не нужен, если он не отличается от основного
\email
    {author@site.ru; co-author@site.ru;  co-author@site.ru}
\thanks
    {Работа выполнена при
     %частичной
     финансовой поддержке РФФИ, проекты \No\ \No 00-00-00000 и 00-00-00001.}
\organization
    {$^1$Организация, адрес; $^2$Организация, адрес}
\abstract
    {Данный текст является шаблоном статьи, подаваемой для публикации в журнале <<Машинное обучение и анализ данных>>.

    Аннотация описывает основную цель работы,
    особенности предлагаемого подхода и~основные результаты.
    Сведения, содержащиеся в заглавии статьи, не должны повторяться в тексте авторского резюме.
    В аннотации не должно быть ссылок на литературу и, по возможности, формул.
	
	Также необходимо представить расширенную структурированную аннотацию на английском языке объемом 200--300 слов.	
	Английская аннотация может не быть дословным переводом русского текста и должна быть написана хорошим английским языком.
	
	В титульном заголовке необходимо указать полный, официально принятый, переводной вариант названия организации.
	Указывать нужно только ту часть названия, которая относится к понятию юридического лица,
	не вписывая названий кафедры, лаборатории или другого структурного подразделения внутри организации.
	Необходимо указать полный юридический адрес, или, как минимум, город и страну.
 	
 	При выборе ключевых слов основным критерием является их потенциальная ценность для выражения содержания документа или для его поиска.
	В качестве ключевых слов могут использоваться термины из названия, аннотации, вступительной и заключительной части текста статьи.
 	При подборе ключевых слов рекомендуется использовать базовые термины вместе с более сложными, допускается использование повторов и синонимов.
	Не рекомендуется использование слишком сложных слов, слов в кавычках, слов с запятыми.
	По возможности следует применять слова в основной форме именительного падежа единственного числа.
	Рекомендуемое количество ключевых слов~-- 5-7, количество слов внутри ключевой фразы~-- не более 3.
	
\bigskip
\noindent
\textbf{Ключевые слова}: \emph {ключевое слово; ключевое слово; еще ключевые слова, разделенные <<;>>}
}

\titleEng
	[JMLDA paper template] % краткое название; не нужно, если полное название влезает в~колонтитул
    {Machine Learning and Data Analysis journal paper template}
\authorEng
	[F.\,S.~Author] % список авторов (не более трех) для колонтитула; не нужен, если основной список влезает в колонтитул
	{F.\,S.~Author, F.\,S.~Co-Author, and F.\,S.~Name} % основной список авторов, выводимый в оглавление
    [F.\,S.~Author$^1$, F.\,S.~Co-Author$^2$, and F.\,S.~Name$^{1, 2}$] % список авторов, выводимый в заголовок; не нужен, если он не отличается от основного
\thanksEng
    {The research was
     %partially
    	 supported by the Russian Foundation for Basic Research (grants 00-00-0000 and 00-00-00001).
    }
\organizationEng
    {$^1$Organization, address; $^2$Organization, address}
\abstractEng
    {This is the template of the paper submitted to the journal ``Machine Learning and Data Analysis''.
		
	\noindent
	The title should be concise and informative. Titles are often used in information-retrieval systems. Avoid abbreviations and formulae where possible.
	Please clearly indicate the last names and initials of each author and check that all names are accurately spelled. Present the authors' affiliation
	addresses where the actual work was done.
	Provide the full postal address of each affiliation, including the country name and, if available, the
	e-mail address of each author.
	Provide only institutional affiliation, department/division affiliation are not required.

	\noindent
	A concise and factual abstract is required.
	The purpose of the abstract is to provide a summary~of the paper enabling the reader to decide whether or not to read the full text.
    	The abstract should state briefly the purpose of the research, the principal results and major conclusions.
    	An abstract is often presented separately from the article, so it must be able to stand alone.
    	For this reason, References should be avoided, but if essential, then cite the author(s) and year(s).
    	Also, non-standard or uncommon abbreviations should be avoided, but if essential they must be defined at their first mention in the abstract itself.
    	The requirements on the size of the abstract is about 200--300 words.
    	It should be provided in the next structured manner:
	
	\noindent
	\textbf{Background}:	One paragraph about the problem, existent approaches and its limitations.
	
	\noindent
	\textbf{Methods}: One paragraph about proposed method and its novelty.
	
	\noindent
	\textbf{Results}: One paragraph about major properties of the proposed method and experiment results if applicable.
	
	\noindent
	\textbf{Concluding Remarks}: One paragraph about the place of the proposed method among existent approaches.
		
	\noindent
	Immediately after the abstract, provide 5-7 keywords, avoiding general and plural terms and multiple concepts (avoid, for example, ``and'', ``of'').
	Use keywords that are specific and that reflect what is essential about the paper.
	Use keywords from the abstract, introduction and conclusion.
	These keywords will be used for indexing purposes.
		
	\noindent
    	\textbf{Keywords}: \emph{keyword; keyword; more keywords, separated by ``;''}}

%данные поля заполняются редакцией журнала
\doi{10.21469/22233792}
\receivedRus{01.01.2017}
\receivedEng{January 01, 2017}

\maketitle
\linenumbers

\section{Введение}
После аннотации, но перед первым разделом,
располагается введение, включающее в себя
описание предметной области,
обоснование актуальности задачи,
краткий обзор известных результатов.

\section{Название раздела}
Данный документ демонстрирует оформление статьи,
подаваемой в электронную систему подачи статей \url{http://jmlda.org/papers} для публикации в журнале <<Машинное обучение и анализ данных>>.
Более подробные инструкции по~стилевому файлу \texttt{jmlda.sty} и~использованию издательской системы \LaTeXe\
находятся в~документе \texttt{authors-guide.pdf}.
Работу над статьёй удобно начинать с~правки \TeX-файла данного документа.

Обращаем внимание, что данный документ должен быть сохранен в кодировке~\verb'UTF-8 without BOM'.
Для смены кодировки рекомендуется пользоваться текстовыми редакторами \verb'Sublime Text' или \verb'Notepad++'.

\paragraph{Название параграфа}
Разделы и~параграфы, за исключением списков литературы, нумеруются.

\section{Заключение}
Желательно, чтобы этот раздел был, причём он не~должен дословно повторять аннотацию.
Обычно здесь отмечают, каких результатов удалось добиться, какие проблемы остались открытыми.

\end{document}
